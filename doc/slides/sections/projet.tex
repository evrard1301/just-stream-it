\section{Préparation du projet}

\begin{frame}{Préparation du projet}
  % TODO: Objectif: -> mettre en place le maximum de bonnes pratiques
  \begin{block}{Étapes}
    \begin{itemize}
    \item Collecte des exigences client.
    \item Préparation du \textit{backlog} produit.
      % TODO: introduire les tests
      % TODO: documenter le projet
    \end{itemize}
  \end{block}
\end{frame}

\begin{frame}{Exigences fonctionnelles}  
\end{frame}

\begin{frame}{Exigences non-fonctionnelles}  
\end{frame}

\begin{frame}{Découpage en tâches}  
\end{frame}

\begin{frame}{État du \textit{backlog} produit}  
\end{frame}

\begin{frame}{Tester le \textit{front-end}}
  % Pourquoi tester le front-end ?
  % TDD + Bibliothèques utilisées
  % Plan de test et documentations
\end{frame}

\begin{frame}{Documenter le \textit{front-end}}
  % Outils pour documenter le javascript
\end{frame}

\begin{frame}{Normes et validations}
  % Outils pour valider l'HTML
  % -> extension firefox HTML validator

  % JS guidelines
  % -> https://developer.mozilla.org/en-US/docs/MDN/Guidelines/Code_guidelines/JavaScript

  % JS Linter avec ESLint
\end{frame}
